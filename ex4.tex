\documentclass[a4paper,
  twoside, % to have to sided mode
  headlines=2.1 % number of lines in the heading, increase if you want more
  ]{scrartcl}
\usepackage[
  margin=2cm,
  includefoot,
  footskip=35pt,
  includeheadfoot,
  headsep=0.5cm,
]{geometry}
\usepackage[utf8]{inputenc}
% \usepackage[ngerman]{babel} % uncomment this line instead of the next one when writing in German
\usepackage[english]{babel}
\usepackage[T1]{fontenc}
\usepackage{mathtools}
\usepackage{amssymb}
\usepackage{lmodern}
\usepackage[automark,headsepline]{scrlayer-scrpage}
\usepackage{enumerate}
\usepackage{sgame}
\usepackage{float}
\usepackage{stmaryrd}
\usepackage{ mathrsfs }
\usepackage[protrusion=true,expansion=true,kerning]{microtype} % just looks much nicer with this

% Reset equation after section and subsection
% \counterwithin*{equation}{section}
% \counterwithin*{equation}{subsection}

\newcommand{\yourname}{Jonas Burster \texttt{4953222} \\ \and Niklas Hain \texttt{4122489} \\ \and Sebastian Weber \texttt{5000486}} % Add more authors with the \and command
% \newcommand{\yourname}{Your Name \\ \texttt{123456789} \and Other Name \\ \texttt{987654321}} % Here an example with matriculation numbers. If you use newlines adjust the command \headingname (don't use new lines there)
\newcommand{\headingname}{\yourname} % Names as they appear in the heading. For example use only surnames when too long
% \newcommand{\headingname}{Dein Name, Anderer Name} % This is an exmaple of a nicely formatted list
\newcommand{\lecture}{Game Theory}
\newcommand{\sheetnum}{1} % Sheet number
\author{\yourname}
\title{\vspace{-20mm}\lecture}
\subtitle{Exercise Sheet 3}
% \subtitle{Übungsblatt \sheetnum} % German version
\date{} % If you want a date add it here (simply use \today for the current date)

% Page heading definitions
\pagestyle{scrheadings}
\setkomafont{pagehead}{\normalfont}
\lohead{\lecture\\\headingname}
\lehead{\lecture\\\headingname}
\rohead{Exercise Sheet 2}
\rehead{Exercise Sheet 2}
% \rohead{Übungsblatt \sheetnum} % German version
% \rehead{Übungsblatt \sheetnum} % German version

\begin{document}
\maketitle
\subsection*{Exercise 4.1 a - Sebastian}
Since A and B for both players are in the support ($supp(\alpha_1)=supp(\alpha_2)=\lbrace A, B \rbrace$), we get the following linear program:
\begin{align}
    u = -10 \alpha_A + \beta(B)\\
    u = -\beta(B)\\
    \beta(A) +\beta(B) = 1\\
    v=-10\alpha(A)+\alpha(B)\\
    v = - \alpha(B)\\
    \alpha(A) + \alpha(B) = 1
\end{align}
Solving it, results in:
\begin{align}
    \beta(A) =\frac{1}{6}\\
    \beta(B) = \frac{5}{6}\\
    \alpha(A) = \frac{1}{6}\\
    \alpha(B) = \frac{1}{6}\\
\end{align}
With this we obtain the mixed Nash Equilibrium of $(\frac{-5}{6}, \frac{-5}{6})$. 

\subsection*{Exercise 4.1 a - Jonas}
Since A and B for both players are in the support ($supp(\alpha_1)=supp(\alpha_2)=\lbrace A, B \rbrace$), we get the following linear program:
\begin{align}
    \alpha_1 -10 \beta(A) + \beta(B)\\
    u = -\beta(B)\\
    \beta(A) +\beta(B) = 1\\
    v=-10\alpha(A)+\alpha(B)\\
    v = - \alpha(B)\\
    \alpha(A) + \alpha(B) = 1
\end{align}
Solving it, results in:
\begin{align}
    \beta(A) =\frac{1}{6}\\
    \beta(B) = \frac{5}{6}\\
    \alpha(A) = \frac{1}{6}\\
    \alpha(B) = \frac{1}{6}\\
\end{align}
With this we obtain the mixed Nash Equilibrium of $(\frac{-5}{6}, \frac{-5}{6})$. 

\subsection*{b -Sebastian}
\begin{align}
    \Omega = \lbrace x,y,z\rbrace\\
    \mathscr{P}_1=\lbrace \lbrace x,y\rbrace,\lbrace z\rbrace \rbrace\\
    \mathscr{P}_2=\lbrace \lbrace x \rbrace,\lbrace y,z\rbrace \rbrace\\
    \pi(x)=\pi(y)=\pi(z)\\
    \sigma_1(x)=\sigma(y)=B\\
    \sigma_1(z)=A\\
    \sigma_2(x)=A\\
    \sigma_2(y)=\sigma(z)=B
\end{align}
Therefore we get a Payoff of (0, 0) and this fulfills the condition correlated NE > mixed NE: \begin{align}
     (0,0)>(\frac{-5}{6}, \frac{-5}{6})
\end{align}
\end{document}