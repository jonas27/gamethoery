\documentclass[a4paper,
  twoside, % to have to sided mode
  headlines=2.1 % number of lines in the heading, increase if you want more
  ]{scrartcl}
\usepackage[
  margin=2cm,
  includefoot,
  footskip=35pt,
  includeheadfoot,
  headsep=0.5cm,
]{geometry}
\usepackage[utf8]{inputenc}
% \usepackage[ngerman]{babel} % uncomment this line instead of the next one when writing in German
\usepackage[english]{babel}
\usepackage[T1]{fontenc}
\usepackage{mathtools}
\usepackage{amssymb}
\usepackage{lmodern}
\usepackage[automark,headsepline]{scrlayer-scrpage}
\usepackage{enumerate}
\usepackage{sgame}
\usepackage{float}
\usepackage[protrusion=true,expansion=true,kerning]{microtype} % just looks much nicer with this

% Reset equation after section and subsection
% \counterwithin*{equation}{section}
% \counterwithin*{equation}{subsection}

\newcommand{\yourname}{Jonas Burster \texttt{4953222} \\ \and Niklas Hain \texttt{4122489} \\ \and Sebastian Weber \texttt{5000486}} % Add more authors with the \and command
% \newcommand{\yourname}{Your Name \\ \texttt{123456789} \and Other Name \\ \texttt{987654321}} % Here an example with matriculation numbers. If you use newlines adjust the command \headingname (don't use new lines there)
\newcommand{\headingname}{\yourname} % Names as they appear in the heading. For example use only surnames when too long
% \newcommand{\headingname}{Dein Name, Anderer Name} % This is an exmaple of a nicely formatted list
\newcommand{\lecture}{Game Theory}
\newcommand{\sheetnum}{1} % Sheet number
\author{\yourname}
\title{\vspace{-20mm}\lecture}
\subtitle{Exercise Sheet 2}
% \subtitle{Übungsblatt \sheetnum} % German version
\date{} % If you want a date add it here (simply use \today for the current date)

% Page heading definitions
\pagestyle{scrheadings}
\setkomafont{pagehead}{\normalfont}
\lohead{\lecture\\\headingname}
\lehead{\lecture\\\headingname}
\rohead{Exercise Sheet 2}
\rehead{Exercise Sheet 2}
% \rohead{Übungsblatt \sheetnum} % German version
% \rehead{Übungsblatt \sheetnum} % German version

\begin{document}
\maketitle
\subsection*{2.1}
\begin{game}{3}{3}[Player~1][Player~2]
& $X$ & $Y$ & $Z$\\
$A$ &(1,1) &(2,1) &(0,2)\\
$B$ &(1,1) &(0,1) &(2,2)\\
$C$ &(2,2) &(1,0) &(1,1)
\end{game}\\

Strategy Y is strictly dominated by Z:\\
\begin{game}{3}{2}[Player~1][Player~2]
& $X$ &  $Z$\\
$A$ &(1,1) &(0,2)\\
$B$ &(1,1) &(2,2)\\
$C$ &(2,2) &(1,1)
\end{game}\\

Strategy A is strictly dominated by C:\\
\begin{game}{2}{2}[Player~1][Player~2]
& $X$ &  $Z$\\
$B$ &(1,1) &(2,2)\\
$C$ &(2,2) &(1,1)
\end{game}\\

Check which strategy profiles are a  Nash Equilibrium:\\
$(B,X)$: No, because Player 2 could play $Z$ instead.\\
$(B,Z)$: Yes.\\
$(C,X)$: Yes.\\
$(C,Z)$: No, because Player 2 could play $X$ instead.\\
$\Rightarrow$ Nash Equilibrium's are: $(B,Z)$ and $(C,X)$

\subsection*{2.2a}
A nash equilibrium is as followed defined:
\begin{equation} 
u_i(a^*) \geq u_i(a_{-i}^*,a_i)
\end{equation}
Therefore we obtain for player 1:
\begin{equation} 
u_i(a_{-i}^*, a_i^*) \geq u_i(a_{-i}^*,a_i)
\end{equation}
In a zero sum game the payoff function of player 1 is related to players 2:
\begin{equation} 
u_i = -u_{-1}
\end{equation}  
If the payoffs for player 1 are changed, the payoffs for player 2 change as well that G' results again in a zero sum game. \\
If we increase the payoffs in G', the nash equilibria will stay the highest payoff until another payoff is higher. Therefore player 1 never gets lower payoff than he got in the Nash equilibria of G.

\subsection*{2.2b}
The game $G$ is a strategic game with $G = \langle N, (A_i)_{i \in N},(u_i)_{i \in N} \rangle$. The game $G'$ results from $G$ by elimination of an arbitrary strategy.\\
The Nash equilibria $a^*$ of $G$ is defined as:
\begin{equation} 
u_i(a_{-i}^*, a_i^*) \geq u_i(a_{i-i}^*, a_i^+)
\end{equation}
Assume $a'$ is Nash equilibrium in $G'$ and
\begin{align}
    u_i(a') > u_i(a^*) \label{eq:ui'gui}
\end{align}
Then $a^*$ can't be a nash equilibria in $G$. Because 
\begin{align}
u_i(a^*) >= u_i(a^*_{-i},a_i) \text{ } \forall a_i \in A_i
\end{align}
does not satisfy for $a'$ because of equation \ref{eq:ui'gui}.

\subsection*{2.3}
Calculate the partial derivative for both players and find max for own action:

For 1:
\begin{equation} 
\max_{{a_1}}U_1=a_2-2a_1\stackrel{!}{=}0
\end{equation}
\begin{equation} 
a_1=0.5a_2
\end{equation}

For 2:
\begin{equation}
\max_{{a_2}}U_2=1-0.5a_1-2a_2\stackrel{!}{=}0\\
\end{equation}
\begin{equation} 
a_2=0.5-0.25a_1
\end{equation}
For a nash equilibria both players actions need to be maxed out. If we substitute $a_1$ in (10) with (8) we get:
\begin{equation}
    a_2 & = \dfrac{1}{2} - \dfrac{1}{4} \cdot \dfrac{1}{2} a_2
\end{equation}
\begin{equation}
    \Leftrightarrow a_2 & = \dfrac{4}{9}
\end{equation}
For $a_1$ we re-substitute $a_2$ in (8):
\begin{align}
    a_1 = \dfrac{1}{2} \cdot \dfrac{4}{9} = \dfrac{2}{9}
\end{align}
So we have a nash equilibria: $(\dfrac{2}{9}, \dfrac{4}{9})$. This is the only one as all of the above equations only have one solution.
\end{document}